\chapter{Testkapitel}
\section{foo}
\blindmathpaper
\section{bar}

Beispiel table:
\begin{table}[H]
    \begin{center}
        \begin{tabular}{ c c c }
            cell1 & cell2 & cell3 \\
            cell4 & cell5 & cell6 \\
            cell7 & cell8 & cell9
        \end{tabular}
        \caption{beispiel table}
    \end{center}
\end{table}

\blindmathpaper
\section{fuzz}

Das \Gls{glossar} wird verwendet.

\blindmathpaper
