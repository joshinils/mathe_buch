\documentclass[openany, oneside, a4paper]{book}

\usepackage[ngerman]{babel}
\usepackage[utf8]{inputenc} % ueoeaess

\usepackage{fancyhdr}
\usepackage{blindtext}
\blindmathtrue

\usepackage{float}

\usepackage{glossaries}
\glsenablehyper
\makenoidxglossaries
\glstoctrue

\parindent0pt

\newglossaryentry{glossar}
{
    name=Glossar,
    description={Ein Glossar ist eine Liste von Wörtern mit beigefügten Bedeutungserklärungen oder Übersetzungen. Als Anhang (Addendum) eines Werkes wird ein Glossar auch als Wörterverzeichnis bezeichnet, ein eigenständiges Glossar als Wörterbuch}
}


\usepackage{hyperref}
\usepackage[all]{hypcap}    %for going to the top of an image when a figure reference is clicked

\usepackage{graphicx}

\usepackage{ifthen}
\newboolean{release}
\setboolean{release}{true}

\usepackage{geometry}
\geometry{
    a4paper,
    total={170mm,257mm},
    left=20mm,
    top=20mm,
}

\ifthenelse{\boolean{release}}{
}{
    \usepackage{showframe}
    \usepackage{layout}
}

\begin{document}
    \ifthenelse{\boolean{release}}{
    }{
        \begin{center}
            \vspace*{20pt}
            \Large
            THIS IS NOT THE RELEASE VERSION!
            \vspace*{20pt}
        \end{center}
        \begin{center}
            \layout
        \end{center}
    }

    \frontmatter
    \pagestyle{plain}
    \include{./front/front}

    \tableofcontents

    \mainmatter
    \pagestyle{fancy}
    \chapter{Testkapitel}
\section{foo}
\blindmathpaper
\section{bar}
\label{section_bar}
\blindmathpaper
\begin{figure}[H]
    \centering
    \includegraphics[width=.25\paperwidth, height=.5\paperwidth, keepaspectratio]{./graphics/empty.png}
    \label{figEmpty}
    \caption{placeholder image}
\end{figure}

\section{fuzz}
\blindmathpaper
\\

Siehe Abb. \ref{figEmpty}
in Section \ref{section_bar}
\\

\blindmathpaper

    \chapter{Testkapitel}
\section{foo}
\blindmathpaper
\section{bar}

Beispiel table:
\begin{table}[H]
    \begin{center}
        \begin{tabular}{ c c c }
            cell1 & cell2 & cell3 \\
            cell4 & cell5 & cell6 \\
            cell7 & cell8 & cell9
        \end{tabular}
        \caption{beispiel table}
    \end{center}
\end{table}

\blindmathpaper
\section{fuzz}

Das \Gls{glossar} wird verwendet.

\blindmathpaper


    \backmatter
    \listoftables
    \listoffigures
    \include{./back/appendix}
    \include{./back/back}

    \printnoidxglossaries
\end{document}