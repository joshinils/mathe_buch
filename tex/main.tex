\documentclass[openany, oneside, a4paper]{book}

\usepackage[ngerman]{babel}
\usepackage[utf8]{inputenc} % ueoeaess
\usepackage[autostyle=true,german=quotes]{csquotes}
\MakeOuterQuote{"}

\usepackage{fancyhdr}
\usepackage{blindtext}
\blindmathtrue

\usepackage{float}

\usepackage{glossaries}
\glsenablehyper
\makenoidxglossaries
\glstoctrue

\parindent0pt

\newglossaryentry{glossar}
{
    name=Glossar,
    description={Ein Glossar ist eine Liste von Wörtern mit beigefügten Bedeutungserklärungen oder Übersetzungen. Als Anhang (Addendum) eines Werkes wird ein Glossar auch als Wörterverzeichnis bezeichnet, ein eigenständiges Glossar als Wörterbuch}
}


\usepackage{hyperref}
\usepackage[all]{hypcap}    %for going to the top of an image when a figure reference is clicked

\usepackage{graphicx}

\usepackage{ifthen}
\newboolean{release}
\setboolean{release}{true}

\usepackage{geometry}
\geometry{
    a4paper,
    total={170mm,257mm},
    left=20mm,
    top=20mm,
}

\usepackage{adjustbox}
\usepackage{amsmath}
\usepackage{amssymb}
\usepackage{nicefrac}

% https://tex.stackexchange.com/a/128066/202560
\newcommand{\eqname}[1]{\tag*{\stackbox{(\theequation)\\#1}}}% Tag equation with name
\usepackage{xparse}
\NewDocumentCommand{\mcomm}{ O{0pt} O{1em} }{\hskip #2 plus #1 minus #2 \vert\;}

\usepackage{cancel}

\ifthenelse{\boolean{release}}{
}{
    \usepackage{showframe}
    \usepackage{layout}
}

\newcommand{\R}{ {\mathbb R} }
\newcommand{\Z}{ {\mathbb Z} }
\newcommand{\C}{ {\mathbb C} }
\newcommand{\N}{ {\mathbb N} }
\newcommand{\e}{ {\,\mathrm e\,} }
\newcommand{\imag}{\/{\mathrm i}\/}
\newcommand{\jmag}{\/{\mathrm j}\/}
\newcommand{\1}{ {\mathds{1}} }
\newcommand{\abs}[1]{\lvert#1\rvert}
\newcommand{\norm}[1]{\left\lVert#1\right\rVert}
\newcommand{\xt}{\tilde{x}}
\newcommand{\dotleq}{\dot{\leq}}
\newcommand{\m}{\hphantom{-} }
\newcommand{\dashfill}[1]{\vspace{11pt}\def\dashfill{\cleaders\hbox{#1}\hfill}\hbox to \hsize{\dashfill\hfil}\vspace{11pt}}
\newcommand{\scdot}{\!\cdot\!}
\newcommand{\sig}{\text{signum}}
\newcommand{\Hess}[1]{\mathop{\mathrm{H}}\nolimits#1}
\newcommand{\laplace}[1]{\mathop{{}\Delta}\nolimits#1}
\newcommand{\td}{\/\,\mathrm{d}}
\newcommand{\kommentar}[1]{}
\newcommand{\Real}[1]{\mathrm{Re}\left\{#1\right\}}
\newcommand{\Imag}[1]{\mathrm{Im}\left\{#1\right\}}
\newcommand{\Spur}[1]{\mathrm{Spur}\!\left(#1\right)}
\newcommand{\skalarprod}[2]{\left\langle#1;#2\right\rangle}
\newcommand{\scalarprod}[2]{\left\langle#1;#2\right\rangle}
\newcommand{\kreuzprod}[2]{#1\times#2}
\newcommand{\crossprod}[2]{\kreuzprod{#1}{#2}}

\begin{document}
    \ifthenelse{\boolean{release}}{
    }{
        \begin{center}
            \vspace*{20pt}
            \Large
            THIS IS NOT THE RELEASE VERSION!
            \vspace*{20pt}
        \end{center}
        \begin{center}
            \layout
        \end{center}
    }

    \frontmatter
    \pagestyle{plain}
    \include{./front/front}

    \tableofcontents

    \mainmatter
    \pagestyle{fancy}
    \chapter{Differentialrechnung}
\section{Ableitungen}

def: Die Ableitung \(f'(x)\) einer Funktion gibt die Steigung dieser an einer Stelle \(x\) zurück:
\begin{figure}[H]
    \centering
    \fbox{\includegraphics[page=1, height=\textheight, width=.7\linewidth, keepaspectratio, trim=3cm 19cm 6cm 18cm, clip]{./graphics/Ableitungen-1.pdf}}
    \label{figEmpty}
    \caption{Ableitungen-1.pdf Seite 1}
\end{figure}
\begin{equation}
    \text{Steigung} = \frac{\Delta y}{\Delta x}
\end{equation}

Wie können wir die Steigung bestimmen?
\begin{figure}[H]
    \centering
    \fbox{\includegraphics[page=2, height=\textheight, width=.7\linewidth, keepaspectratio, trim=0cm 29.5cm 8cm 7cm, clip]{./graphics/Ableitungen-1.pdf}}
    \label{figEmpty}
    \caption{Ableitungen-1.pdf Seite 2}
\end{figure}
Steigung von \(f(x)\):
\begin{equation}
    f'(x) = \lim_{h\to 0}\frac{f(x+h)-f(x)}{h}
    \eqname{Differentialquotient}
\end{equation}

Beispiel:
\begin{align}
    f(x) &= x^2\\
    \Rightarrow f'(x) &= \lim_{h\to 0}\frac{f(x+h)-f(x)}{h}\\
    &=\lim_{h\to 0}\frac{(x+h)^2-(x)^2}{h}\\
    &=\lim_{h\to 0}\frac{x^2+2xh + h^2 -x^2}{h} \mcomm \text{Binomische Formel}\\
    &=\lim_{h\to 0}\frac{2xh + h^2}{h} \mcomm x^2-x^2=0\\
    &=\lim_{h\to 0}\frac{h \cdot(2x + h)}{h} \mcomm h\text{ ausklammern}\\
    &=\lim_{h\to 0}\frac{\cancel{h} \cdot(2x + h)}{\cancel{h}} \mcomm h\text{ kürzen}\\
    &=\lim_{h\to 0}(2x + h)\\
    &=2x + 0 \mcomm h=0 \text{ einsetzen}\\
    \Rightarrow f'(x) &= 2x
\end{align}
\begin{equation}
    \boxed{f(x) = x^2 \Rightarrow f'(x)=2x}
\end{equation}

Da das sehr aufwendig ist, gibt es einfachere Regeln um Ableitungen zu bestimmen:

Ableitungsregel:\\
1. Potenzregel:
\begin{align}
    f(x) &= ax^n \quad,\; n \in \mathbb{R}\setminus\{0\}\\ %
    \Rightarrow f'(x) &= n \cdot ax^{n-1} \mcomm \text{"vom Exponenten fällt ein \(n\) vorne dran"}
\end{align}

\begin{figure}[H]
    \centering
    \fbox{\includegraphics[page=4, height=\textheight, width=.7\linewidth, keepaspectratio, trim=0cm 0cm 0cm 0cm, clip]{./graphics/Ableitungen-1.pdf}}
    \label{figEmpty}
    \caption{Ableitungen-1.pdf Seite 4}
\end{figure}

\begin{figure}[H]
    \centering
    \fbox{\includegraphics[page=5, height=\textheight, width=.7\linewidth, keepaspectratio, trim=0cm 0cm 0cm 0cm, clip]{./graphics/Ableitungen-1.pdf}}
    \label{figEmpty}
    \caption{Ableitungen-1.pdf Seite 5}
\end{figure}

\begin{figure}[H]
    \centering
    \fbox{\includegraphics[page=6, height=\textheight, width=.7\linewidth, keepaspectratio, trim=0cm 0cm 0cm 0cm, clip]{./graphics/Ableitungen-1.pdf}}
    \label{figEmpty}
    \caption{Ableitungen-1.pdf Seite 6}
\end{figure}

\begin{figure}[H]
    \centering
    \fbox{\includegraphics[page=7, height=\textheight, width=.7\linewidth, keepaspectratio, trim=0cm 0cm 0cm 0cm, clip]{./graphics/Ableitungen-1.pdf}}
    \label{figEmpty}
    \caption{Ableitungen-1.pdf Seite 7}
\end{figure}

\begin{figure}[H]
    \centering
    \fbox{\includegraphics[page=8, height=\textheight, width=.7\linewidth, keepaspectratio, trim=0cm 0cm 0cm 0cm, clip]{./graphics/Ableitungen-1.pdf}}
    \label{figEmpty}
    \caption{Ableitungen-1.pdf Seite 8}
\end{figure}

    %\chapter{Testkapitel}
\section{foo}
\blindmathpaper
\section{bar}
\label{section_bar}
\blindmathpaper
\begin{figure}[H]
    \centering
    \includegraphics[width=.25\paperwidth, height=.5\paperwidth, keepaspectratio]{./graphics/empty.png}
    \label{figEmpty}
    \caption{placeholder image}
\end{figure}

\section{fuzz}
\blindmathpaper
\\

Siehe Abb. \ref{figEmpty}
in Section \ref{section_bar}
\\

\blindmathpaper

    %\chapter{Testkapitel}
\section{foo}
\blindmathpaper
\section{bar}

Beispiel table:
\begin{table}[H]
    \begin{center}
        \begin{tabular}{ c c c }
            cell1 & cell2 & cell3 \\
            cell4 & cell5 & cell6 \\
            cell7 & cell8 & cell9
        \end{tabular}
        \caption{beispiel table}
    \end{center}
\end{table}

\blindmathpaper
\section{fuzz}

Das \Gls{glossar} wird verwendet.

\blindmathpaper


    \backmatter
    \listoftables
    \listoffigures
    \include{./back/appendix}
    \include{./back/back}

    \printnoidxglossaries
\end{document}